%% -*- mode: LaTeX -*-
%%
%% Copyright (c) 1997, 1998, 1999 The University of Utah and the Computer
%% Systems Laboratory at the University of Utah (CSL).
%%
%% This file is part of Flick, the Flexible IDL Compiler Kit.
%%
%% Flick is free software; you can redistribute it and/or modify it under the
%% terms of the GNU General Public License as published by the Free Software
%% Foundation; either version 2 of the License, or (at your option) any later
%% version.
%%
%% Flick is distributed in the hope that it will be useful, but WITHOUT ANY
%% WARRANTY; without even the implied warranty of MERCHANTABILITY or FITNESS
%% FOR A PARTICULAR PURPOSE.  See the GNU General Public License for more
%% details.
%%
%% You should have received a copy of the GNU General Public License along with
%% Flick; see the file COPYING.  If not, write to the Free Software Foundation,
%% 59 Temple Place #330, Boston, MA 02111, USA.
%%

%%%%%%%%%%%%%%%%%%%%%%%%%%%%%%%%%%%%%%%%%%%%%%%%%%%%%%%%%%%%%%%%%%%%%%%%%%%%%%%

Flick's runtimes provide the definitions and functions that are necessary in
order for you to compile Flick-generated stubs into working code.  For
instance, Flick's runtime header files provide the definitions of the macros
used in generated stubs.  Flick's runtime libraries provide functions to manage
object references, interact with the underlying IPC layer, and so on.

Flick's current runtimes were written to provide the minimum functionality
required by Flick-generated code.  The runtimes are not intended to be highly
flexible or composable or even complete; for instance, Flick's IIOP runtime for
C stubs has minimal CORBA ORB functionality.  The current set of runtimes is
simply sufficient to get clients and servers talking to one another.  Future
versions of Flick will include more complete runtimes, or will include more
code generators that work with runtimes developed by third parties.


%%%%%%%%%%%%%%%%%%%%%%%%%%%%%%%%%%%%%%%%%%%%%%%%%%%%%%%%%%%%%%%%%%%%%%%%%%%%%%%

\section{The Runtime Libraries and Headers}
\label{sec:The Runtime Libraries and Headers}

Flick's runtime libraries are contained under the \filename{runtime/libraries}
directory of the Flick distribution:

\begin{filenamelist}
  \item[libflick-iiop.a] The IIOP runtime library for C stubs, to be using with
  stubs generated by Flick's IIOP back end, \program{flick-c-pbe-iiop}.

  \item[libflick-suntcp.a] The ONC/TCP runtime library, to be used with stubs
  generated by Flick's ONC/TCP back end, \program{flick-c-pbe-sun}.

  \item[libflick-mach3mig.a] The Mach~3 runtime library, to be used with stubs
  generated by Flick's Mach~3 back end, \program{flick-c-pbe-mach3mig}.

  \item[libflick-trapeze.a] The Trapeze runtime library, to be used with stubs
  generated by Flick's Trapeze back end, \program{flick-c-pbe-trapeze}.
\end{filenamelist}

The runtime libraries for three of Flick's back ends are distributed separately
and are not further described in this manual:

\begin{commandlist}
  \item[flick-c-pbe-iiopxx] As previously described, this back end generates
  C++ stubs that work with TAO version~\taoversion{}, the real-time ORB from
  Washington University in St.\ Louis.  You will need a copy of TAO (from
  \taourl{} or \taosrcurl{}) in order to run the code produced by Flick's
  IIOP/C++ back end.

  \item[flick-c-pbe-khazana] The Khazana runtime will be included as part of
  the Khazana distribution.  See \khazanaurl{} for information about Khazana.

  \item[flick-c-pbe-fluke] The runtime library and header files for Fluke stubs
  are included as part of the Fluke operating system.  See \flukeurl{} for
  information about Fluke.
\end{commandlist}

Flick's runtime header files are contained in the \filename{runtime/headers}
directories of your Flick source tree \emph{and} your Flick build tree.  While
most of Flick's runtime headers are located in the Flick source tree, two of
Flick's runtime header files are generated by Flick's \program{configure}
script, and are therefore located in the \emph{object tree} that you create
when you build the Flick tools.  To compile Flick-generated stubs, you must
make sure that both the source- and object-tree \filename{runtime/headers}
directories are contained in your C compiler's system include path.
Alternately, the Flick installation process (described in
Section~\ref{sec:Installing Flick}) will place all of the required header files
in a single place on your system.

The rest of this chapter describes Flick's IIOP and ONC/TCP runtimes in detail.
Chapter~\ref{cha:Putting It Together: A Phonebook Application} shows how to use
each of these runtimes in a simple phonebook application.


%%%%%%%%%%%%%%%%%%%%%%%%%%%%%%%%%%%%%%%%%%%%%%%%%%%%%%%%%%%%%%%%%%%%%%%%%%%%%%%

\section{The IIOP Runtime}
\label{sec:The IIOP Runtime}

%% -*- mode: LaTeX -*-
%%
%% Copyright (c) 1997, 1998, 1999 The University of Utah and the Computer
%% Systems Laboratory at the University of Utah (CSL).
%%
%% This file is part of Flick, the Flexible IDL Compiler Kit.
%%
%% Flick is free software; you can redistribute it and/or modify it under the
%% terms of the GNU General Public License as published by the Free Software
%% Foundation; either version 2 of the License, or (at your option) any later
%% version.
%%
%% Flick is distributed in the hope that it will be useful, but WITHOUT ANY
%% WARRANTY; without even the implied warranty of MERCHANTABILITY or FITNESS
%% FOR A PARTICULAR PURPOSE.  See the GNU General Public License for more
%% details.
%%
%% You should have received a copy of the GNU General Public License along with
%% Flick; see the file COPYING.  If not, write to the Free Software Foundation,
%% 59 Temple Place #330, Boston, MA 02111, USA.
%%

%%%%%%%%%%%%%%%%%%%%%%%%%%%%%%%%%%%%%%%%%%%%%%%%%%%%%%%%%%%%%%%%%%%%%%%%%%%%%%%

The IIOP runtime is a \emph{very minimal} ORB-like system, implementing only
the functions that are required to implement basic client/server communication.
The current runtime should be sufficient, however, to allow Flick-generated
stubs to communicate with any other implementation of IIOP, including programs
running on top of other ORBs.
%
The following sections describe how to use Flick's IIOP runtime.  A complete
application based on the IIOP runtime is described in Section~\ref{sec:The
CORBA Phonebook in C}.


%%%%%%%%%%%%%%%%%%%%%%%%%%%%%%%%%%%%%%%%%%%%%%%%%%%%%%%%%%%%%%%%%%%%%%%%%%%%%%%

\subsection{Server Command Line Options}
\label{subsec:Server Command Line Options}

When an IIOP server starts, the Flick-generated \cfunction{main} function calls
\cfunction{CORBA_ORB_init} to initialize the IIOP runtime and start the ORB\@.
Flick's runtime searches for and processes the following command line
arguments:

\begin{optionlist}
  \item[-ORBipaddr~\optionarg{ipaddress}]
  %
  (Optional.)  Specify the network interface for the server.  This option is
  useful if your machine has multiple network interfaces.  If unspecified, the
  server will use the host's primary network interface.

  \item[-ORBid~\optionarg{name}]
  %
  (Optional.)  Use \optionarg{name} as the server's ORB identifier; see Section
  14.26.1 of the CORBA~2.0 specification.  This option is generally not used.
  The default ORB identifier is ``FlickORB-{}-''.
  %
  % Old text, apparently no longer true, removed for Flick 1.1 release ---
  %
  % if you specify a different name, then client programs will need to take
  % special steps in order to create object references (because
  % \cfunction{CORBA_ORB_string_to_object} won't be able to create object
  % references by itself).  See Section~\ref{subsec:Creating Object References}
  % below for details.

  \item[-OAport~\optionarg{portnumber}]
  %
  (Required.)  Specify the port that the server will use.  There is no default
  value; the port number must be explicitly given on the command line.

  \item[-OAid~\optionarg{name}]
  %
  (Optional.)  Use \optionarg{name} as the server's BOA (Basic Object Adapter)
  identifier; see Section~14.26.2 of the CORBA~2.0 specification.  Like
  \option{-ORBid}, this option is rarely used.  The default BOA identifier is
  ``FlickBOA-{}-''.
  %
  % Old text, apparently no longer true, removed for Flick 1.1 release ---
  %
  % and if you specify a different value, then client programs will need to
  % take special steps in order to create object references.
\end{optionlist}


%%%%%%%%%%%%%%%%%%%%%%%%%%%%%%%%%%%%%%%%%%%%%%%%%%%%%%%%%%%%%%%%%%%%%%%%%%%%%%%

\subsection{Creating Object Implementations}
\label{subsec:Creating Object Implementations}

After initializing the ORB and the BOA (done for you by the Flick-generated
\cfunction{main} function), a server must create one or more object
implementations (object instances) that will be available to service client
requests.  These implementations are created by a function called
\cfunction{register_objects} which you must write as part of your server's
application code.  The prototype for this function is:

\begin{verbatim}
   void register_objects(CORBA_ORB, CORBA_BOA, int argc, char **argv,
                         CORBA_Environment *);
\end{verbatim}

\noindent The \cfunction{main} function generated by Flick will call your
version of \cfunction{register_objects} in order to initialize your server's
objects.  In general, your \cfunction{register_objects} function should parse
the given command line in order to create its objects.  Your function should
call \cfunction{CORBA_BOA_create} to create each of its object
instances:\footnote{Flick's version of \cfunction{CORBA_BOA_create} differs
slightly from what is specified by Section~8.2 of the CORBA~2.0
specification.  This will eventually be fixed, probably by updating the runtime
to conform to the newer Portable Object Adapter (POA) specification.}

\begin{verbatim}
   CORBA_Object CORBA_BOA_create(CORBA_BOA boa,
                                 CORBA_ReferenceData *obj_key,
                                 const char *obj_type, FLICK_SERVER obj_impl,
                                 CORBA_Environment *ev);
\end{verbatim}

\noindent The first argument should be the \ctype{CORBA_BOA} that was passed to
your \cfunction{register_objects} function.  The second argument is a pointer
to an octet sequence that names the new object instance.  This name must be
unique across all object instances.  As is usual, the sequence is described by
a C structure containing three fields: \cidentifier{_buffer},
\cidentifier{_length}, and \cidentifier{_maximum}.  The third argument is a
string that identifies the type (class) of the object; this string may be in
any format you choose.  (Flick does not parse the object type string; Flick
uses the string only when creating and parsing object references.)  The fourth
argument to \cfunction{CORBA_BOA_create} is a pointer to the Flick-generated
dispatch function for the object interface; as far as Flick's IIOP runtime is
concerned, this is what determines the object's class.  The name of the server
dispatch function is usually the name of the interface concatenated with the
suffix ``\cfunction{_server}''.  The fifth and final argument is a pointer to a
\ctype{CORBA_Environment}; your \cfunction{register_objects} function should
check the environment after each call to \cfunction{CORBA_BOA_create} to see if
an error has occurred.

It is useful for your \cfunction{register_objects} function to create any
additional data structures that your server will need in order to represent its
objects.  For instance, the phonebook application in Section~\ref{sec:The CORBA
Phonebook in C} creates a list of structures to represent its phonebooks; each
structure is keyed by a \ctype{CORBA_ReferenceData} and contains the data for a
single phonebook object.  The server work functions later use
\cfunction{CORBA_BOA_get_id} to locate the data for a phonebook object, mapping
from a CORBA object reference to the object's \ctype{CORBA_ReferenceData}, and
then using that name to locate the phonebook object's data.

Your \cfunction{register_objects} function should raise an exception if it is
unable to initialize your server's objects.  If no exception is raised, then
the Flick-generated \cfunction{main} server function will begin to process
requests on the registered objects.


%%%%%%%%%%%%%%%%%%%%%%%%%%%%%%%%%%%%%%%%%%%%%%%%%%%%%%%%%%%%%%%%%%%%%%%%%%%%%%%

\subsection{Creating Object References}
\label{subsec:Creating Object References}

A client can create an object reference with the following function:

\begin{verbatim}
   CORBA_Object CORBA_ORB_string_to_object(CORBA_ORB orb, CORBA_char *name,
                                           CORBA_Environment *ev);
\end{verbatim}

The name can be either a stringified Interoperable Object Reference (IOR) or a
URL-style name of the form
``\option{iiop:1.0//\emph{hostname}:\emph{port}/\emph{objecttype}/\emph{objectname}}''.\footnote{The
idea for URL-style object references was adopted from SunSoft's reference IIOP
implementation, which is available at
\urlftp{ftp://ftp.omg.org/pub/interop/iiop.tar.Z}.  Flick's URL-style
references differ from those supported by the SunSoft IIOP implementation in
that Flick's references include the object type.}  The latter form is generally
preferred because it is easier to type.  Whenever a Flick-based server creates
an object instance, the server prints both the IOR and URL-style name of the
new object.

Alternately, a client can create an object reference by going through the
following steps:

\begin{enumerate}
  \item Initialize an ORB, using \cfunction{CORBA_ORB_init}.

  \item Create a BOA, using \cfunction{CORBA_ORB_BOA_init}.

  \item Create the object reference by calling \cfunction{CORBA_BOA_create}
  with the appropriate \ctype{CORBA_ReferenceData} (key) value.
\end{enumerate}

This procedure will find the corresponding object instance within the given ORB
and create an appropriate object reference.  Of course,
\cfunction{CORBA_ORB_string_to_object} is generally simpler for clients to use.


%%%%%%%%%%%%%%%%%%%%%%%%%%%%%%%%%%%%%%%%%%%%%%%%%%%%%%%%%%%%%%%%%%%%%%%%%%%%%%%

\subsection{Using Object References}
\label{subsec:Using Object References}

Once a client has obtained an object reference, it may use the reference to
invoke operations on the object as defined by the object's IDL interface.
Flick's IIOP runtime provides these additional functions for manipulating
object references:

{\footnotesize
\begin{list}{}{}
  \item%
    {\cprototype{CORBA_char *CORBA_ORB_object_to_string(CORBA_ORB, CORBA_Object, CORBA_Environment *);}}%
    \hspace{0pt}\\
  Create and return a stringified Interoperable Object Reference (IOR).

  \item%
    {\cprototype{CORBA_char *CORBA_ORB_object_to_readable_string(CORBA_ORB, CORBA_Object, CORBA_Environment *);}}%
    \hspace{0pt}\\
  Create and return a human-readable, URL-style stringified object reference.
  (This is not a standard CORBA function.)

  \item%
    {\cprototype{CORBA_Object CORBA_ORB_string_to_object(CORBA_ORB, CORBA_char *, CORBA_Environment *);}}%
    \hspace{0pt}\\
  Create and return an object reference from a stringified object reference
  (IOR or URL).

  \item%
    {\cprototype{CORBA_ReferenceData *CORBA_BOA_get_id(CORBA_BOA, CORBA_Object, CORBA_Environment *);}}%
    \hspace{0pt}\\
  Return the corresponding key for the given object reference.

  \item%
    {\cprototype{CORBA_boolean CORBA_Object_is_nil(CORBA_Object, CORBA_Environment *);}}%
    \hspace{0pt}\\
  Return true if the object reference is nil.

  \item%
    {\cprototype{CORBA_Object CORBA_Object_duplicate(CORBA_Object, CORBA_Environment *);}}%
    \hspace{0pt}\\
  Duplicate an object reference.

  \item%
    {\cprototype{void CORBA_Object_release(CORBA_Object, CORBA_Environment *);}}%
    \hspace{0pt}\\
  Release an object reference.

  \item%
    {\cprototype{CORBA_unsigned_long CORBA_Object_hash(CORBA_Object, CORBA_unsigned_long, CORBA_Environment *);}}%
    \hspace{0pt}\\
  Return a value suitable for use in a hash table.  The value is not guaranteed
  unique, but good enough to implement an efficient hash table with the number
  of buckets indicated by the second argument.

  \item%
    {\cprototype{CORBA_boolean CORBA_Object_is_equivalent(CORBA_Object, CORBA_Object, CORBA_Environment *);}}%
    \hspace{0pt}\\
  Return true if the two object references refer to the same object instance.
\end{list}
}% End `\footnotesize'.


%%%%%%%%%%%%%%%%%%%%%%%%%%%%%%%%%%%%%%%%%%%%%%%%%%%%%%%%%%%%%%%%%%%%%%%%%%%%%%%

%% End of file.




%%%%%%%%%%%%%%%%%%%%%%%%%%%%%%%%%%%%%%%%%%%%%%%%%%%%%%%%%%%%%%%%%%%%%%%%%%%%%%%

\section{The ONC/TCP Runtime}
\label{sec:The ONC/TCP Runtime}

%% -*- mode: LaTeX -*-
%%
%% Copyright (c) 1997 The University of Utah and the Computer Systems
%% Laboratory at the University of Utah (CSL).
%%
%% This file is part of Flick, the Flexible IDL Compiler Kit.
%%
%% Flick is free software; you can redistribute it and/or modify it under the
%% terms of the GNU General Public License as published by the Free Software
%% Foundation; either version 2 of the License, or (at your option) any later
%% version.
%%
%% Flick is distributed in the hope that it will be useful, but WITHOUT ANY
%% WARRANTY; without even the implied warranty of MERCHANTABILITY or FITNESS
%% FOR A PARTICULAR PURPOSE.  See the GNU General Public License for more
%% details.
%%
%% You should have received a copy of the GNU General Public License along with
%% Flick; see the file COPYING.  If not, write to the Free Software Foundation,
%% 59 Temple Place #330, Boston, MA 02111, USA.
%%

%%%%%%%%%%%%%%%%%%%%%%%%%%%%%%%%%%%%%%%%%%%%%%%%%%%%%%%%%%%%%%%%%%%%%%%%%%%%%%%

Flick's ONC/TCP runtime was designed so that it would be possible for users to
write an ONC~RPC-based client or server, and then compile the application code
to use either Flick-generated stubs or \program{rpcgen}-generated stubs.  This
requirement introduced a few oddities into Flick's ONC/TCP runtime but can be
very helpful in demonstrating the speed gained by using Flick-generated stubs.
(If you are curious, the files \filename{test/programs/sun/sunstat-work.c} and
\filename{test/programs/sun/sunstat-use.c} in the Flick distribution show how
to write code that is compatible with both Flick and \program{rpcgen}.)


%%%%%%%%%%%%%%%%%%%%%%%%%%%%%%%%%%%%%%%%%%%%%%%%%%%%%%%%%%%%%%%%%%%%%%%%%%%%%%%

\subsection{Starting a Server}
\label{subsec:Starting a Server}

Flick's ONC/TCP back end creates a \cfunction{main} function that automatically
registers the interfaces described in the original IDL file.  (In an ONC~RPC
IDL file, an interface is a program-version pair.)  There are no objects to
create or register; the server itself is the ``object'' in ONC~RPC\@.  The
Flick-generated \cfunction{main} function then executes the server main loop,
receiving requests and sending replies.

If the Flick-generated \cfunction{main} function is suitable for your
application, then you need do nothing more.  If, however, your server needs to
perform special initialization, you will have to write your own
\cfunction{main} function to set up your application, register your interfaces,
and then run the main server loop.  The rest of this section describes the
procedure for registering your interfaces with Flick's ONC/TCP runtime.

Your \cfunction{main} function must initialize a
\ctype{FLICK_SERVER_DESCRIPTOR} structure for each interface that it wants to
register:

\begin{verbatim}
   typedef struct FLICK_SERVER_DESCRIPTOR {
           unsigned int prog_num;   /* The program number. */
           unsigned int vers_num;   /* The version number. */
           ...
   } FLICK_SERVER_DESCRIPTOR;
\end{verbatim}

\noindent The values of \cidentifier{prog_num} and \cidentifier{vers_num}
fields would normally be the program and version numbers listed in your
original IDL file.  After filling in the \ctype{FLICK_SERVER_DESCRIPTOR}
structure, your server must call \cfunction{flick_server_register} to register
the interface with the runtime:

\begin{verbatim}
   /* Returns 1 for success or 0 for failure. */
   int flick_server_register(FLICK_SERVER_DESCRIPTOR, FLICK_SERVER);
\end{verbatim}

\noindent The first argument is the structure containing the program and
version numbers that was previously filled in.  The second argument is a
pointer to the Flick-generated server dispatch function for the interface.  The
name of this function is generally the name of the ONC~RPC program, followed by
an underscore and the version number of the interface:
``\texttt{\emph{program}\_\emph{version}}''.

After your \cfunction{main} function has registered all of its interfaces, it
must call the \cfunction{flick_server_run} function.  This function takes no
arguments and should never return.  If it does, there has been a fatal error.


%%%%%%%%%%%%%%%%%%%%%%%%%%%%%%%%%%%%%%%%%%%%%%%%%%%%%%%%%%%%%%%%%%%%%%%%%%%%%%%

\subsection{Connecting to a Server}
\label{subsec:Connecting to a Server}

When a server is run, it registers itself with the RPC portmapper facility.
Client processes can connect to a running server simply by specifying the host
machine of the server and the program and version numbers of the desired
interface.  To connect to a server, a client process fills out a
\ctype{FLICK_SERVER_LOCATION} structure:

\begin{verbatim}
   typedef struct FLICK_SERVER_LOCATION {
           char *server_name;      /* The host machine.   */
           unsigned int prog_num;  /* The program number. */
           unsigned int vers_num;  /* The version number. */
   } FLICK_SERVER_LOCATION;
\end{verbatim}

\noindent Once this structure is initialized, the client calls
\cfunction{flick_client_create} to connect to the server:

\begin{verbatim}
   /* Returns 1 for success or 0 for failure. */
   int flick_client_create(CLIENT *, FLICK_SERVER_LOCATION);
\end{verbatim}

\noindent The first argument must point to a \ctype{CLIENT} structure; that
structure will be initialized by the call to \cfunction{flick_client_create}.


%%%%%%%%%%%%%%%%%%%%%%%%%%%%%%%%%%%%%%%%%%%%%%%%%%%%%%%%%%%%%%%%%%%%%%%%%%%%%%%

\subsection{Using the Server Connection}
\label{subsec:Using the Server Connection}

Once a client has established a connection to a server, it must pass the
initialized \ctype{CLIENT} data to the Flick-generated stub functions in order
to make RPCs.  When the client is ready to close its connection to the server,
it should call \cfunction{flick_client_destroy}:

\begin{verbatim}
   void flick_client_destroy(CLIENT *);
\end{verbatim}


%%%%%%%%%%%%%%%%%%%%%%%%%%%%%%%%%%%%%%%%%%%%%%%%%%%%%%%%%%%%%%%%%%%%%%%%%%%%%%%

\subsection{Using the Service Request Data (\texttt{svc\_req})}
\label{subsec:Using the Service Request Data}

The second argument to an ONC~RPC-style server work function is generally a
pointer to a \ctype{struct svc_req}, a structure that describes the context of
the current RPC: the program, version, and operation numbers, the client's
credentials, and so on.  When writing a server work function, you must include
this argument in your function's parameter list; see
Section~\ref{subsec:ONCRPC:Server Functions} for examples.

However, beware: \emph{Flick's ONC/TCP runtime does not currently initialize
the structure that is passed to your work function!}  Currently,
Flick-generated stubs include this parameter simply to be prototype-compatible
with \program{rpcgen}-generated code.  In a future version of Flick, the
correct \ctype{svc_req} data will be provided to server work functions.


%%%%%%%%%%%%%%%%%%%%%%%%%%%%%%%%%%%%%%%%%%%%%%%%%%%%%%%%%%%%%%%%%%%%%%%%%%%%%%%

%% End of file.




%%%%%%%%%%%%%%%%%%%%%%%%%%%%%%%%%%%%%%%%%%%%%%%%%%%%%%%%%%%%%%%%%%%%%%%%%%%%%%%

%% End of file.

