%% -*- mode: LaTeX -*-
%%
%% Copyright (c) 1997, 1999 The University of Utah and the Computer Systems
%% Laboratory at the University of Utah (CSL).
%%
%% This file is part of Flick, the Flexible IDL Compiler Kit.
%%
%% Flick is free software; you can redistribute it and/or modify it under the
%% terms of the GNU General Public License as published by the Free Software
%% Foundation; either version 2 of the License, or (at your option) any later
%% version.
%%
%% Flick is distributed in the hope that it will be useful, but WITHOUT ANY
%% WARRANTY; without even the implied warranty of MERCHANTABILITY or FITNESS
%% FOR A PARTICULAR PURPOSE.  See the GNU General Public License for more
%% details.
%%
%% You should have received a copy of the GNU General Public License along with
%% Flick; see the file COPYING.  If not, write to the Free Software Foundation,
%% 59 Temple Place #330, Boston, MA 02111, USA.
%%

%%%%%%%%%%%%%%%%%%%%%%%%%%%%%%%%%%%%%%%%%%%%%%%%%%%%%%%%%%%%%%%%%%%%%%%%%%%%%%%

%% Define a command for commenting-out text.
%%
\long\def\com#1{}

%% Define some commands for defining URLs.  These commands create URLs that
%% turn into hypertext links when we process this document with `TeX4ht'.
%% If we're not producing HTML, we need to provide `\HCode' as a no-op.
%%
\providecommand{\HCode}[1]{}
\newcommand{\urlhttp}{%
  \begingroup%
  \def\UrlLeft##1\UrlRight{\HCode{<A HREF="##1">}##1\HCode{</A>}}%
  \urlstyle{tt}%
  \Url}
\newcommand{\urlmailto}{%
  \begingroup%
  \def\UrlLeft##1\UrlRight{\HCode{<A HREF="mailto:##1">}##1\HCode{</A>}}%
  \urlstyle{tt}%
  \Url}
\newcommand{\urlftp}{%
  \begingroup%
  \def\UrlLeft##1\UrlRight{\HCode{<A HREF="##1">}##1\HCode{</A>}}%
  \urlstyle{tt}%
  \Url}

%% Define some Flick-related URLs.
%%
\urldef{\flickurl}{\urlhttp}{http://www.cs.utah.edu/flux/flick/}
\urldef{\flickemail}{\urlmailto}{flick@cs.utah.edu}
\urldef{\flickbugsemail}{\urlmailto}{flick-bugs@cs.utah.edu}
\urldef{\flukeurl}{\urlhttp}{http://www.cs.utah.edu/flux/}
\urldef{\khazanaurl}{\urlhttp}{http://www.cs.utah.edu/flux/}
\urldef{\fluxurl}{\urlhttp}{http://www.cs.utah.edu/flux/}

\urldef{\taourl}{\urlhttp}{http://www.cs.wustl.edu/~schmidt/TAO.html}
\urldef{\taosrcurl}{\urlftp}{ftp://ace.cs.wustl.edu/pub/ACE/}
\urldef{\trapezeurl}{\urlhttp}{http://www.cs.duke.edu/ari/trapeze/}

%% What version of TAO do we support?
%%
\newcommand{\taoversion}{\mbox{1.0}}

%% Define some commands to typeset the names of files, C functions, etc.
%%
\newcommand{\filename}{\begingroup \Url}

\newcommand{\cfunction}{\begingroup \Url}
\newcommand{\cprototype}{\begingroup \Url}
\newcommand{\ctype}{\begingroup \Url}
\newcommand{\cidentifier}{\begingroup \Url}

\newcommand{\idl}{\begingroup \Url}

\newcommand{\program}{\begingroup \Url}

% `\option' is not a URL because we want to be able to invoke macros with an
% option, e.g., `\optionarg' and `\optionor'.
\newcommand{\option}[1]{\texttt{#1}}

%% Define an environment for lists of command line options.  This `optionlist'
%% environment was adapted from the `Mentry' example on page 65 of _The_LaTeX_
%% _Companion_ by Goossens et al.
%%
\newcommand{\optionlistlabel}[1]%
  {\raisebox{0pt}[1ex][0pt]%
            {\makebox[\labelwidth][l]%
                     {\parbox[t]{\labelwidth}%
                             {\raggedright\hspace{0pt}\option{#1}}%
                     }%
            }%
  }
\newcommand{\optionarg}[1]{\texttt{\emph{#1}}}
\newcommand{\oroption}{\textrm{or}}
\newenvironment{optionlist}%
  {\begin{list}{}%
         {\renewcommand{\makelabel}{\optionlistlabel}%
          %
          % Provide a command for padding the height of the text.  Required
          % when the height of the text is less than the height of the item
          % label.
          \newcommand{\optionpad}{\\\mbox{}}%
          %
          % Set our various list lengths.
          \setlength{\labelwidth}{2.0in}%
          \setlength{\leftmargin}{2.4in}%
          \setlength{\parsep}{1ex plus 0.3ex}%
          \setlength{\partopsep}{0pt}%
          \setlength{\itemsep}{0pt}%
         }%
  }%
  {\end{list}}

%% Define an environment for lists of filenames.  This is essentially identical
%% to the `optionlist' environment above.  If the filename is too weird, this
%% won't work; see the comments in `url.sty'.
%%
\newcommand{\filenamelistlabel}[1]%
  {\raisebox{0pt}[1ex][0pt]%
            {\makebox[\labelwidth][l]%
                     {\parbox[t]{\labelwidth}%
                             {\raggedright\hspace{0pt}\filename{#1}}%
                     }%
            }%
  }
\newenvironment{filenamelist}%
  {\begin{list}{}%
         {\renewcommand{\makelabel}{\filenamelistlabel}%
          \newcommand{\filenamepad}{\\\mbox{}}%
          \setlength{\labelwidth}{2.0in}%
          \setlength{\leftmargin}{2.4in}%
          \setlength{\parsep}{1ex plus 0.3ex}%
          \setlength{\partopsep}{0pt}%
          \setlength{\itemsep}{0pt}%
         }%
  }%
  {\end{list}}

%% Define an environment for lists of commands (i.e., command lines).  This is
%% essentially identical to the `optionlist' environment above.  Note that this
%% is explicit *not* a `programlist': `\program' typesets its argument as a URL
%% whereas this environment does not.
%%
\newcommand{\commandlistlabel}[1]%
  {\raisebox{0pt}[1ex][0pt]%
            {\makebox[\labelwidth][l]%
                     {\parbox[t]{\labelwidth}%
                             {\raggedright\hspace{0pt}\texttt{#1}}%
                     }%
            }%
  }
\newcommand{\commandarg}[1]{\texttt{\emph{#1}}}
\newcommand{\orcommand}{\textrm{or}}
\newenvironment{commandlist}%
  {\begin{list}{}%
         {\renewcommand{\makelabel}{\commandlistlabel}%
          \newcommand{\commandpad}{\\\mbox{}}%
          \setlength{\labelwidth}{2.0in}%
          \setlength{\leftmargin}{2.4in}%
          \setlength{\parsep}{1ex plus 0.3ex}%
          \setlength{\partopsep}{0pt}%
          \setlength{\itemsep}{0pt}%
         }%
  }%
  {\end{list}}

%% Redefine `\maketitle' to include our copyright notice.  This is almost
%% identical to the `titlepage' version of `\maketitle' from the LaTeX2e
%% standard `book.cls' file.
%%
\makeatletter
\renewcommand{\maketitle}{\begin{titlepage}%
  \let\footnotesize\small
  \let\footnoterule\relax
  \let \footnote \thanks
  \null\vfil
  \vskip 60\p@
  \begin{center}%
    {\LARGE \@title \par}%
    \vskip 3em%
    {\large
     \lineskip .75em%
      \begin{tabular}[t]{c}%
        \@author
      \end{tabular}\par}%
      \vskip 1.5em%
    {\large \@date \par}%       % Set date in \large size.
  \end{center}\par
  \vfill                        % <<< new
  \noindent%                    % <<< new
  {\small%                      % <<< new
   \@copyrightnotice@begin%     % <<< new
   \@copyrightnotice%           % <<< new
   \@copyrightnotice@end}%      % <<< new
  \@thanks
  \vfil\null
  \end{titlepage}%
  \setcounter{footnote}{0}%
  \global\let\thanks\relax
  \global\let\maketitle\relax
  \global\let\@thanks\@empty
  \global\let\@author\@empty
  \global\let\@date\@empty
  \global\let\@title\@empty
  \global\let\title\relax
  \global\let\author\relax
  \global\let\date\relax
  \global\let\and\relax
}
\makeatother


%%%%%%%%%%%%%%%%%%%%%%%%%%%%%%%%%%%%%%%%%%%%%%%%%%%%%%%%%%%%%%%%%%%%%%%%%%%%%%%

%% End of file.

